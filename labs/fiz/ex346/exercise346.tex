\documentclass[a4paper,12pt]{article}

\usepackage{latexsym}
\usepackage[utf8]{inputenc}
\usepackage[T1]{fontenc}
\usepackage{graphicx}
\usepackage{amsmath}
\usepackage{float}% If comment this, figure moves to Page 2
\usepackage[hidelinks]{hyperref}
\usepackage{lipsum}

\author{Krzysztof~Palka and Dominik~Odrowski}
\date{April 25, 2013}

\title{\textsc{Exercise} 346 \\ Examination of permittivity of ferroelectricity} 

\addtolength{\textwidth}{2.5cm}
\addtolength{\hoffset}{-1.25cm}

\begin{document}

    \maketitle

    \begin{abstract}
        This report presents measurement of the permittivity of ferroelectricity material using parallel-plate capacitor.
    \end{abstract}

    \section{Introduction}
    The aim of this exercise was to determine, using experimental setup mainly based on capacitor, the Curie-Weiss temperature, dependency of permittivity from temperature in paraelectric and ferroelectric phase and in general to acquaint with properties of ferroelectric materials. 

    \section{Theory and measurement}

    

    Electric permittivity can be determinate by measurement capacitance of capacitor $C$, which plates stick to plate cut form examined substance. As thickness of examined material is much less then it's area $S$ we can calculate capacitance form formula for capacitor with 2 parallel plates. Transformation of this equation lets us to calculate electric permittivity form equation:    
    \begin{equation}
        \epsilon = \frac{d C}{S \epsilon_0} \label{eq:epsilon}
    \end{equation}

    Near temperature of phase transition form ferroelectric to paraelectric permittivity can be described by Curie-Weiss low:    
    \begin{equation}
        \epsilon = \frac{K}{T-T_c} \label{eq:curie}
    \end{equation}
    where $K$ is Curie-Weiss constant. According to that inverse of permittivity is proportional to temperature  
    \begin{equation}
        \frac{1}{\epsilon} = \frac{1}{K} T - \frac{T_c}{K} \label{eq:curie2}
    \end{equation}
    It allows us to determine constans $T_c$ and $K$, lets assume following factors:  
    \begin{equation}
        \frac{1}{\epsilon} = mT + b \label{eq:curie3}
    \end{equation}
    So respectively:    
    \begin{equation}
        K = \frac{1}{m} \label{eq:k}
    \end{equation}
    and 
    \begin{equation}
        T_c = - K b \label{eq:tc}
    \end{equation}
    

    \section{Results}

    For calculation of $\epsilon$ we have used following parameters read form experimental set-up: $S = 89 \mathrm{mm}^2$, $d = 1.1$ mm. To read capacitance we have added additional capacitance form electric connections, meters etc. This value also was given and was equal approximately 48 pF.  


    \begin{table}[H]
        \begin{center}
            \caption{Measured temperature $T$, capacitance $C$ and calculated permittivity $\epsilon$ and inverse of permittivity $\epsilon^{-1}$}
            \label{tab:vn}
    
            \begin{tabular}{|c|c|c|c||c|c|c|c|}
                \hline
                $T$ [K] & $C$ [pF] & $\epsilon$ & $\epsilon^{-1}$ [$10^{-3}$] &
                $T$ [K] & $C$ [pF] & $\epsilon$ & $\epsilon^{-1}$ [$10^{-3}$] 
                \\ \hline
342.5 & 127.2 & 242.0 & 4.133 & 320.5 & 2197.0 & 3100.4 & 0.323\\
340.5 & 130.4 & 246.4 & 4.059 & 318.5 & 1573.1 & 2238.8 & 0.447\\
338.5 & 144.5 & 265.9 & 3.762 & 316.5 & 1304.6 & 1868.0 & 0.535\\
336.5 & 150.0 & 273.4 & 3.657 & 314.5 & 1136.6 & 1636.0 & 0.611\\
334.5 & 165.9 & 295.4 & 3.385 & 312.5 & 1009.4 & 1460.3 & 0.685\\
332.5 & 184.5 & 321.1 & 3.114 & 310.5 & 911.9 & 1325.7 & 0.754\\
330.5 & 217.0 & 366.0 & 2.732 & 308.5 & 833.8 & 1217.8 & 0.821\\
328.5 & 255.2 & 418.7 & 2.388 & 306.5 & 765.8 & 1123.9 & 0.890\\
326.5 & 325.5 & 515.8 & 1.939 & 304.5 & 715.6 & 1054.6 & 0.948\\
324.5 & 436.9 & 669.7 & 1.493 & 302.5 & 672.7 & 995.3 & 1.005\\
322.5 & 883.5 & 1286.4 & 0.777 & 301.1 & 636.3 & 945.0 & 1.058\\
                \hline
            \end{tabular}
        \end{center}
    \end{table}


    \begin{figure}[H]
    \begin{center}
        \includegraphics[width=0.7\textwidth]{epsilon}
        \caption{Graph of permittivity for measured temperature}
        \label{fig:epsilon}
    \end{center}
    \end{figure}

    \begin{figure}[H]
    \begin{center}
        \includegraphics[width=0.7\textwidth]{epsilon-1}
        \caption{Graph of inverse of permittivity for measured temperature}
        \label{fig:epsilon-1}
    \end{center}
    \end{figure}

    Usage of least square regression for paraelectric state of of material gives factors $m = (-3.6 \pm 0.2) \cdot 10^{-2}$ and $b = (1.19 \pm 0.03) \cdot 10^{-1} \mathrm{K}$. Substituting those values to equations \ref{eq:k} and \ref{eq:tc}, we have obtained results:

    \begin{displaymath}
        K = \frac{1}{-3.6 \cdot 10^{-2}} \approx -2.7 \cdot 10^3    
    \end{displaymath}

    and 

    \begin{displaymath}
        T_c = -1 \cdot -2.7 \cdot 10^3 \cdot 1.19 \cdot 10^{-1} \mathrm{K} \approx 321 \mathrm{K}
    \end{displaymath}

    Propagation of error can be calculated form equations: 

    \begin{equation}
        \Delta K = K \frac{\Delta a}{a}
    \end{equation}
    
    \begin{equation}
        \Delta T_c = T_c \left( \frac{\Delta K}{K} + \frac{\Delta b}{|b|} \right)
    \end{equation}
    
    So complete results are as following

    \begin{displaymath}
        K = (-2.7 \pm 0.2) \cdot 10^3
    \end{displaymath}
    
    \begin{displaymath}
        T_c = 321 \pm 1 K
    \end{displaymath}
    


    \section{Conclusions}

    As we can observe on graph \ref{fig:epsilon-1} temperature of phase transition calculated from linear approximation by least squares method seems to be true in real results. Unfortunately we couldn't find any ferroelectric with such Curie temperature. The nearest was made of arsen and mangan (MnAs) and had temperature 318 K \cite{res}, so either our results was not as accurate as it follows from uncertainty or we had examined different material. There was couple factors that inflated on results, for example precision of termometr, location of it sensor, additional capacitance of set-up. Using formula for capacitor with 2 parallel plates could affect result, but not in significant way. Other reason could be not sufficient resolution of measurement, 2 K in this case could led to bigger errors.  

    \begin{thebibliography}{9}


        \bibitem{res}\emph{Handbuch der Physik} (1966) F. Keffer. New York: Springer-Verlag. Cite available online at: \url{http://hyperphysics.phy-astr.gsu.edu/hbase/tables/curie.html}. Accessed April 29, 2013.

    \end{thebibliography}

\end{document}
